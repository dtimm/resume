\documentclass[letterpaper,12pt]{article}
\usepackage{hyperref}
\usepackage{fullpage}
\urlstyle{same}
\author{David A Timm}
\title{Resume}
\setcounter{secnumdepth}{-1}
\begin{document}
\pagenumbering{gobble}
\begin{center}
\section{David A Timm}

E: \href{mailto:timmmeister@gmail.com?subject=You%27re%20hired!}
{\nolinkurl{timmmeister@gmail.com}}
T: \mbox{(303) 652-7177}
\end{center}

\begin{center}
\subsection{Technical Skills}
\end{center}

\begin{itemize}
\item Learning new things quickly (I learned \LaTeX{} just for this!)
\item C++, C\textsuperscript{{$\sharp$}}.NET, MFC, Python, Java, GNU tools, MS Access and MySQL
\item General problem isolation and troubleshooting (I'm awesome at this)
\end{itemize}

\begin{center}
\subsection{Experience}
\end{center}

\begin{flushleft}
\textbf{Lead Developer, AutoSPRINK for Revit}: 01/15--Present M.E.P.CAD Inc.
\end{flushleft}

Pipes. All of the pipes. I write pipe-specific software for all of the world's
piping needs. Really, there are a few projects that I work on, but C++ MFC
stuff to route, alter, do complicated analysis of, and generally deal with
pipes is our bread and butter. In six months, I have mostly done two things:
A) refactored a clunky C\textsuperscript{{$\sharp$}}.NET plugin for AutoDesk Revit,
increasing speed by a factor of ten and B) shipped.

\begin{flushleft}
\textbf{Site Visit Tech}: 05/13--01/15 Denver Public Schools
\end{flushleft}

I roamed from town to town, helping those in need\ldots{} I supported schools in
the deployment, use, and repair of any technology a student may lay their hands
on. Smart boards, computers of all stripes, clickers, nComputing, printers, and
even a few electric pencil sharpeners.

\begin{flushleft}
\textbf{Hotline Tech II}: 05/11--05/13 Denver Public Schools
\end{flushleft}

I was first/second line tech support for staff of Denver Public Schools.
Financial and HR software, wireless connections, Mac, PC, all that fun stuff.
My record for tickets processed in a day still stands at 232.

\begin{flushleft}
\textbf{Student}: 02/10--04/11 Colorado School of Trades
\end{flushleft}

Gunsmithing? Really!? I spent fourteen months completing a program that
encompassed troubleshooting, repairing, and building (from blueprints or to
purpose) parts for firearms. It has helped my ability to visualize problems
and directly taught me a lot of very interesting skills. There were many
resources available to students to complete their work, and as a student, I
was expected to take advantage of them. The curriculum takes students through
basic fabrication, machining, woodworking, and design/function courses,
covering special tools and techniques as they apply to the trade. I completed
the program with a 3.3 GPA.
\end{document}
