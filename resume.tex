\documentclass[letterpaper,12pt]{article}
\usepackage{hyperref}
\usepackage{fullpage}
\urlstyle{same}
\author{David A Timm}
\title{Resume}
\setcounter{secnumdepth}{-1}
\begin{document}
\pagenumbering{gobble}
\begin{center}
\section{David A Timm}

E: \href{mailto:timmmeister@gmail.com?subject=You%27re%20hired!}
{\nolinkurl{timmmeister@gmail.com}}
T: \mbox{(303) 652-7177}
\end{center}

\begin{center}
\subsection{Technical Skills}
\end{center}

\begin{itemize}
\item Learning new things quickly (I just finished a 12-month Udacity degree in 2 months!)
\item C++, C\textsuperscript{{$\sharp$}}.NET, MFC, Python, Java, GNU tools, MS Access and MySQL
\item General problem isolation and troubleshooting (I'm awesome at this)
\end{itemize}

\begin{center}
\subsection{Experience}
\end{center}

\begin{flushleft}
\textbf{Machine Learning Nanodegree}: 05/16--08/16 Udacity
\end{flushleft}

I wanted to stretch my wings a little, so I decided to take some courses on
Udacity. I dipped my feet into the data science coursework and decided to go
for broke in machine learning. Numpy, scikit-learn, and TensorFlow were all
covered in varying depths. The course provided a broad overview of machine
learning techniques, and the final project gave me an excellent opportunity to
dive into deep learning (har har har...).

\begin{flushleft}
\textbf{Development Lead}: 01/15--Present M.E.P.CAD Inc.
\end{flushleft}

Pipes. All of the pipes. I write pipe-specific software for all of the world's
piping needs. M.E.P.CAD has several products for design and testing of fire
sprinkler and alarm systems, and I have worked on all of them. Currently, I'm
the development lead on a project to bring the fire sprinkler world into
Autodesk's Revit. I lead a team of four in product design and execution with
input from fire protection engineers at our headquarters in Nevada.

\begin{flushleft}
\textbf{Technician}: 05/11--01/15 Denver Public Schools
\end{flushleft}

I roamed from town to town, helping those in need\ldots{} I supported schools in
the deployment, use, and repair of any technology a student may lay their hands
on. Smart boards, computers of all stripes, clickers, nComputing, printers, and
even a few electric pencil sharpeners. I started in the internal IT ``hotline,''
but ended as a roaming site-support technician.

\begin{flushleft}
\textbf{Student}: 02/10--04/11 Colorado School of Trades
\end{flushleft}

Gunsmithing? Really!? I spent fourteen months completing a program that
encompassed troubleshooting, repairing, and building (from blueprints or to
purpose) parts for firearms. It has helped my ability to visualize problems
and directly taught me a lot of very interesting skills. There were many
resources available to students to complete their work, and as a student, I
was expected to take advantage of them. The curriculum takes students through
basic fabrication, machining, woodworking, and design/function courses,
covering special tools and techniques as they apply to the trade. I completed
the program with a 3.3 GPA.
\end{document}
